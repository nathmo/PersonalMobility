\section{Introduction}

\subsection{Problem Statement}

Personal vehicles contribute significantly to green house gas emission and are inherently inefficient.\cite{us_epa_carbon_2015} Public transportation systems offer more potential for sustainable mobility, yet they often face challenges in achieving widespread adoption.

 This project seeks to address these issues using a first-principles approach to develop a personal mobility solutions that minimize energy loss, reduce emissions, and better integrate with existing infrastructure using real world data.

\subsection{Objectives of the Project}

This project aims to explore the design space of personal mobility solution. Define a metric for efficiency and the constraint to make the system desirable while significantly more efficient; List the current state of the literature; Finally, study in details the road behavior of the emerging solutions.

\subsection{Methodology Overview}

To tackle this problem, we will follow a  top-down approach using a model defined from first principle and real world measurements. Then, we will incorporate the ergonomics, psychological, social constraint from the users.
The proposed vehicle dynamic behavior will then be studied, followed by a performance analysis.