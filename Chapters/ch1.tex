\section{Introduction}

\subsection{Problem Statement}

Personal car contribute significantly to green house gas emission and are inherently inefficient \cite{us_epa_carbon_2015}. 
City around the world are struggling with both traffic jam and lack of parking space for car.
There is a gap for an intermediate vehicle that combines the efficiency, road/parking use and affordability of a bike with the comfort and utility of a car to replace the majority of trips.

 This project seeks to show that solutions exist to these issues using a first-principles.

\subsection{Objectives of the Project}

This project aims to demonstrate that it is possible for a vehicle that solve the problem to exist and that it is feasible with leveraging current technology. The goal is to prove that this category of vehicle can meet real-world mobility needs, making it a compelling option for replacing most short-distance car trips in Europe.

\subsection{Methodology Overview}

To tackle this problem, a top-down approach using a model defined from first principle and real world measurements is followed. Then, the ergonomic, psychological and social constraint from the users are taken into consideration. A literature review is made to address the existing solutions. Finally,the proposed vehicle's is defined, it's dynamic behavior is studied and a performance comparison with current solution is made.