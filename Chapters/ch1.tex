\section{Introduction}

\subsection{Problem Statement}

Personal vehicles contribute significantly to green house gas emission and are inherently inefficient.\cite{us_epa_carbon_2015} Public transportation systems are essential for sustainable mobility, yet they often face challenges in achieving widespread adoption, particularly in less densely populated areas. In rural regions, lower population density leads to reduced ridership for fixed transit routes and a smaller tax base to fund public transportation systems, limiting its potential as a viable alternative to personal vehicles.\cite{noauthor_barriers_nodate}

 This project seeks to address these issues using a first-principles approach to developing practical and sustainable personal mobility solutions that minimize energy loss, reduce emissions, and better integrate with existing infrastructure.

\subsection{Objectives of the Project}

This project aims to explore the design space of personal mobility solution. Define a metric for efficiency and the constraint to make the system desirable while significantly more efficient. Finally, study in details the road behavior of the emerging solutions.

\subsection{Methodology Overview}

To tackle this problem, we will follow a  top-down approach using a model defined from first principle. Then, we will incorporate the ergonomics and psychological and social constraint from the users.
The proposed vehicle dynamic behavior will then be studied, followed by a performance, economic and legal analysis.