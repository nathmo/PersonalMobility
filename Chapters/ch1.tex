\section{Introduction}

\subsection{Problem Statement}

Personal car contribute significantly to green house gas emission and are inherently inefficient \cite{us_epa_carbon_2015}. 
Cities around the world are struggling with both traffic jam and lack of car parking space.
There is a need for an intermediate vehicle that combines the efficiency, road/parking use and affordability of a bike with the comfort and utility of a car, and would provide the majority of short-distance trips on the road.

\subsection{Objectives of the Project}

The objective of this project is to first analyze the constraints on urban vehicle design from an efficiency, urban and human standpoint.
Secondly, to propose an improved trade-off between efficiency and usability through a vehicle design.
The last goal is then to validate the design viability and behavior via modeling and simulation

\subsection{Methodology Overview}

A top-down approach has been used, which relies on a model defined from first principle and real world measurements. Then, the ergonomic, psychological and social constraints from the users are taken into consideration. To provide a perspective on several existing solutions, a list of literature references is included.  Finally, the proposed vehicle's is defined, its dynamic behavior is studied, and a performance evaluation is conducted.