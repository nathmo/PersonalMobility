\section{User-Centered Design}

While technical performance and environmental impact are often emphasized in vehicle design, user adoption ultimately hinges on satisfying real human needs, real or perceived. This section highlights the limitations of current electric velomobiles and bicycles from a user-centered perspective and explains how our concept seeks to overcome them.

\subsection{Human Elements}

User satisfaction depends on a holistic experience that includes not only mobility but also comfort, safety and convenience.
Multiple studies confirm that key decision factors in choosing a personal vehicle go beyond speed or energy use and include subjective feelings of safety, practicality and comfort. Among these factor here the main one we identified :

\begin{itemize}
    \item \textbf{Feeling visible and protected:}
    One of the most frequently cited concerns among cyclists is the lack of safety feeling, especially in mixed traffic environments \cite{marincek_comparing_2023}. Low-seated vehicles like velomobiles and trikes often exacerbate this issue by limiting eye contact and visibility. In contrast, automotive design trends show a steady increase in H-point height reflecting both consumer preference and perceived improvements in safety and comfort\cite{noauthor_riding_2008}. Higher seating positions offer better road visibility and make occupants feel more secure, which helps explain the popularity of SUVs and crossovers. Applying this principle to lightweight vehicles can enhance user confidence and comfort in urban settings.

    \item \textbf{Physical effort:}
    High or inconsistent effort discourages regular cycling. The PASTA study \cite{castro_physical_2019} showed that e-bikes reduce physical strain while enabling longer trips, helping users maintain consistent daily use. By smoothing effort and lowering peak demands, electric assistance improves accessibility and long-term retention, especially for commuters.

    \item \textbf{Storage and passenger flexibility:} Although data from EU transport surveys show that cars generally carry only one person \cite{ceu_move_study_2022}, occasional cargo (e.g., groceries, luggage) and passengers (children, partner) must be accommodated. Thus, capacity for a second seat or generous cargo area is essential without overshooting and ending up with five seat that add very little marginal utility.

    \item \textbf{Weather and thermal protection:} 
    Exposure to cold air and rain are commonly cited deterrents to cycling and velomobile use. The study by Nankervis \cite{nankervis_effect_1999} confirms that weather and seasonal conditions do influence cycling frequency, especially among less-committed riders, suggesting that improved weather protection and thermal comfort can help support regular use. In warm climates cooling also becomes relevant.
    
    \item \textbf{Cabin privacy and comfort:} 
    Cars provide a personal enclosure that protects occupants from external elements such as weather, noise, and pollution. This enclosed space creates a “personal bubble” that enhances comfort and a sense of security. In contrast, bicycles and velomobiles generally lack such comprehensive protection, which has been identified as one of the drivers behind why many users prefer enclosed vehicles \cite{noauthor_freedom_nodate}.

    \item \textbf{Ease of entry and exit:} The amount of automotive research dedicated to optimizing ingress and egress highlights its importance in vehicle usability and user satisfaction. \cite{roof_height_ingress} \cite{giacomin_analysis_1997} \cite{causse_dynamic_2009}. The popularity of SUVs, in part, stems from their elevated stance, which allows users to enter and exit without crouching is particularly beneficial for older adults.

\end{itemize}

