\section{Innovative Design Concept}

\subsection{Elevating the Vehicle Without Increasing Frontal Area}

Current approach tend to have the vehicle body occupy the whole front area and utilize this space to house the leaning system. Instead, we will focus on an active swing arm design with passive caster wheel. The act of leaning will force the wheel to turn in the wanted direction. This arrangement offer the advantage of occupying less frontal area while allowing to raise and lower the vehicle to both improve visibility in traffic and facilitate entrance and exit.

Make a drawing with the cabin + passenger and wheel. show the design space.

Design 0 : tilting cabin, show the occupied space

design A : 2 front arm along body, passive wheel to turn, one fixed wheel behind

design B : 2 front arm along body, passive wheel to turn, 2 wheel on two arm behind.

design C : 2 front arm sideways crab of body, passive wheel to turn, one fixed wheel behind

design D : 2 front arm sideways crab of body, passive wheel to turn, one fixed wheel behind

design E : 2 front arm sideways crab of body + suspension vertical, passive wheel to turn, one fixed wheel behind

design F : 2 front arm sideways crab of body + suspension vertical, passive wheel to turn, 2 wheel on arm behind

design G : 2 front arm, lateral extension of body + suspension vertical, passive wheel to turn, one fixed wheel behind

design H : 2 front arm lateral extension of body + suspension vertical, passive wheel to turn, 2 wheel on arm behind

design I : 2 front arm lateral extension of body + suspension vertical, two passive wheels to turn behind

design J : 2 front wheel with passive pivot, arm lateral extension of body + suspension vertical and powered motor behind

design H : 1 front wheel with passive pivot. Dual arm lateral extension of body + suspension vertical and powered motor behind

design I : 1 rear wheel with passive pivot. Dual arm lateral extension of body + suspension vertical and powered motor in front


from previous chapter, want to lower frontal area while keeping user eyes high from the ground\\

how can we do so on a kinematic standpoint, and what synergies can we gain from coupling the steering to the leaning ?\\

Use wheel to help control, differential speed\\

3 wheel 1 front -> \\
3 wheel 2 front -> \\
4 wheel         -> \\

Tilting body, \\
plunging wheel, \\
(arm)\\

Free to Roll\\
Free to caster\\
Active control\\

*** steering approach\\
-> skid steer\\
-> ackerman\\
-> swerve wheel + crab steering\\
-> 2 wheel lean to steer (trail + no trail ?)\\
-> other\\
*** geometry and kinematic\\
-> vertical actuator\\
-> arm along vehciule\\
-> arm sideways vehicule\\
-> others ?\\


Wheel, contact patch, function of camber angle on Self-Aligning Torque\\
wheel, gyroscope, \\
wheel, mass and shock absorption \\
Fork Offset, effect on Self-Aligning Torque\\



\subsection{Benefits for Vertical Parking and Accessibility}
talk about the advantage of the selected design, how it could park vertically and what good it would do in city.\\

compare to a normal car, show what would happen if it can park 6 time more. talk about rebound effect risk.

talk about the stability while boarding and exiting, how the vehicle can help to make it a "normal chair" posture then reclining, frontal entry (lateral feel less stable)\\
front entry for more stability.

\subsection{Integration with Public Transport Infrastructure (e.g., Trains)}
possible advantage of loading the vehicle on train for high speed, long distance travel. could beat both car and public transport ? light enough to be lifter by a person  and small enough to occupy roughly the same footprint as a bike.