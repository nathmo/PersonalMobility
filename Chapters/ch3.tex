\section{Design Considerations}

\subsection{Reducing Frontal Area and improving the coefficient of drag}

A typical passenger car offer five seats, while the average occupancy rate for private car remains well below two persons per car \cite{ceu_move_study_2022}.

To reduce the frontal area, we consider a smaller vehicle with one or two seat. Passengers would be one behind the other. This offers the benefit of both reducing the width of the vehicle while also allowing to create a more elongated envelope that improve the coefficient of drag. Furthermore, we can also reduce the effective used height of the vehicle by tilting the seats, allowing to further reduce the frontal area.

show graphically the frontal area of a car and the passenger inside. Front and side view. (graph SUV side + front and proposed seating side + front

show the proposed solution to reduce the space occupied with the tilted chair. show the lying position / human tuboid.

talk about ergonomic and the user FoV, limitation on height reduction (or deport viewing with set of problem like motion sickness)

Talk about ergonomics, how to get in and out
\subsection{Reducing mass and embodied energy}
why are car heavy ? what is energy intensive in a car ?
basically say that to reduce mass we can use lighter material, but we should be careful to not replace them by more intensive one. Wood and PET sheet might actually be an interesting solution.

\subsection{Addressing Visibility and Safety Concerns in Traffic}

why trike and velomobile are not more mainstream (economics and  driving lower than anybody else, more ?)

\subsection{User comfort requirement}
(thermal, noise, water ingress, seating position)
study about what people need in car. we are we sensitive about when travelling / commuting


joystick -> poignet + bas que le coude

nuque -> 4-20°  -> max 15° depuis le 4° ->  total 30°

talk about ergonomy and the user FoV, limitation on height reduction (or deport viewing with set of problem like motion sickness)

https://www.epfl.ch/labs/chili/fr/chili/

\subsection{Narrow Track and Low Vehicle Geometry (literature review)}
Acknowledge existing solution