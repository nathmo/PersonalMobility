\documentclass[a4paper, 11 pt]{article}
\input{General/Preamble} % Loads in the preamble 
\input{General/Settings} % Loads in user defined settings
\input{General/MathCommands} % Loads in user defined math commands

\begin{document}

%%% PAGE DE GARDE
\thispagestyle{empty}
\begin{center}
    \newcommand{\HRule}{\rule{\linewidth}{0.5mm}}
    
    \textsc{\LARGE École Polytechnique Fédérale de Lausanne}\\[0.5cm]
    \includegraphics[width=5cm]{Figures/epfl_red.png}\\[0.5cm]
    \LARGE{\textsc{Robotique, 2024-25}}\\[0.5cm]
    \huge{\textbf{Semester Project}}
    
    \HRule \\[0.4cm]
    {\huge \bfseries A First-Principles Approach to Pragmatic Personal Mobility}\\[0.1cm]
    \HRule \\[0.5cm]
    \LARGE{\textsc{Biorobotics Laboratory} \\
    \text{07.02.2025 - 30.06.2025}}\\[0.5cm]
    
    \vspace{5mm}
    %\includegraphics[scale=0.6]{images/pageDeGarde.png}\\
    \vspace{15mm}
    
    \begin{minipage}{2in} \Large
    \textbf{Author:}\\
    Nathann \textsc{Morand}\\
    Robotics M.Sc.
    \end{minipage}
    \hfill
    \begin{minipage}{2.1in} \Large
    \textbf{Supervisors:}\\
    Mohamed \textsc{Bouri} \\
    \end{minipage}
\end{center}
%%%

%%% ABSTRACT, PROJECT ASSIGNMENT
{ % create local environment
\fancyhead[R]{\textsc{PROJECT ASSIGNMENT, ABSTRACT}}

\newpage
\section*{Abstract}
\addcontentsline{toc}{section}{Abstract}
Urban environments face increasing challenges related to road traffic congestion, limited parking, and the environmental inefficiencies of personal cars. Meanwhile, current cycle-based alternatives such as e-bikes and velomobiles often fail to meet comfort, safety, or usability expectations, limiting their adoption. This project investigates the feasibility of a new vehicle category which bridges that gap: a compact, lightweight, and energy-efficient vehicle concept offering car-like comfort while retaining the legal and spatial advantages of a bicycle.

Using a first-principles approach, we model the energy demands of short-distance urban trips, accounting for both ideal and real-world driving patterns. We identify key parameters such as mass, frontal area, and drivetrain efficiency that govern energy use. We then introduce a novel vehicle design that incorporates an active leaning mechanism, vertical parking capability, and tandem seating, aiming to balance aerodynamic performance, urban compatibility, and user experience.

The vehicle concept is evaluated against existing solutions in terms of efficiency, spatial footprint, and human factors. A comparative simulation study using PyBullet explores the steady-state dynamic stability of various configurations (3-wheel vs. 4-wheel, pivot vs. linear leaning mechanisms) under PID control of the proposed design. Results demonstrate that the proposed design offers superior trade-offs in the three metric defined, albeit at the cost of increased mechanical complexity and the need for active stabilization requiring a complex controller.

This study provides both a physical and behavioral justification for a pragmatic urban mobility solution. It also outlines the path toward prototyping, with considerations for dynamic control and safety.

\newpage
}
%%%
%%% TABLE OF CONTENTS
\newpage
\renewcommand{\contentsname}{Table of Contents} 
{
  \hypersetup{linkcolor=black}
  \tableofcontents
}
\newpage
%%%

% Creates the introduction, starting page numbering
\section{Introduction}

\subsection{Problem Statement}

Personal vehicles contribute significantly to green house gas emission and are inherently inefficient.\cite{us_epa_carbon_2015} Public transportation systems are essential for sustainable mobility, yet they often face challenges in achieving widespread adoption, particularly in less densely populated areas. In rural regions, lower population density leads to reduced ridership for fixed transit routes and a smaller tax base to fund public transportation systems, limiting its potential as a viable alternative to personal vehicles.\cite{noauthor_barriers_nodate}

 This project seeks to address these issues using a first-principles approach to developing practical and sustainable personal mobility solutions that minimize energy loss, reduce emissions, and better integrate with existing infrastructure.

\subsection{Objectives of the Project}

This project aims to explore the design space of personal mobility solution. Define a metric for efficiency and the constraint to make the system desirable while significantly more efficient. Finally, study in details the road behavior of the emerging solutions.

\subsection{Methodology Overview}

To tackle this problem, we will follow a  top-down approach using a model defined from first principle. Then, we will incorporate the ergonomics and psychological and social constraint from the users.
The proposed vehicle dynamic behavior will then be studied, followed by a performance, economic and legal analysis. 
\section{Foundations of Efficiency}
Quantifying the energy demand of a vehicle begins with first principles. This section establishes a framework to understand how physical forces, design choices, and driving behavior jointly determine the energy efficiency of small personal vehicles. Starting from the classical longitudinal force balance, we derive expressions for steady-state cruising power and identify dominant energy losses. We then extend the model to incorporate real-world driving conditions idling, acceleration, braking and the embedded energy associated with mass and manufacturing. By disentangling these contributors, we identify which design decisions yield the most significant reductions in energy consumption, both during operation and over the vehicle’s full lifecycle.

\subsection{Longitudinal Dynamics: A First-Principles Approach}
The longitudinal dynamics of a ground vehicle can be expressed by Newton’s second law:

\begin{figure}[h!]
    \centering
    \includegraphics[width=1\linewidth]{Figures/ch1_ForceAxis.png}
    \caption{Longitudinal car dynamics}
    \label{fig:longcardynamics}
\end{figure}

To model the energy requirements of a vehicle in motion, we begin with the classical longitudinal force balance:

\[
m \cdot a = F_{\text{motor}} - F_{\text{drag}} - F_{\text{roll}} + F_{\text{gravity}}
\]

This equation expresses Newton’s second law applied to a vehicle moving along a slope, where \( m \) is the vehicle mass and \( a \) its acceleration. The right-hand side aggregates all relevant external forces acting on the vehicle in the direction of travel.

Substituting each component force into the equation, we obtain the fully expanded expression:

\[
m \cdot a = F_{\text{motor}} - \frac{1}{2} \rho\, C_d\, A\, v^2 - C_{rr}\, m\, g + m\, g\, \sin\theta
\]

This relation captures the competing effects of propulsion, aerodynamic drag, rolling resistance, and gravitation trough road slope. Each parameter affect directly energy consumption.

\vspace{0.5em}
\noindent
\textbf{Definition of Parameters:}
\begin{align*}
m &:\ \text{vehicle mass [kg]} \\
a &:\ \text{longitudinal acceleration [m/s}^2\text{]} \\
F_{\text{motor}} &:\ \text{force produced by the motor or absorbed by braking [N]} \\
\rho &:\ \text{air density [kg/m}^3\text{]} \\
C_d &:\ \text{aerodynamic drag coefficient [–]} \\
A &:\ \text{frontal area of the vehicle [m}^2\text{]} \\
v &:\ \text{vehicle speed [m/s]} \\
C_{rr} &:\ \text{rolling resistance coefficient [–]} \\
g &:\ \text{gravitational acceleration [m/s}^2\text{]} \\
\theta &:\ \text{road slope angle [rad]} \\
\end{align*}

Note that the gravitational term becomes negative when descending (\(\theta < 0\)) and positive when climbing. While the average gravitational contribution over a round trip cancels out, energy losses due to braking and powertrain inefficiencies remain.

Based of the previous equation, we can define the efficiency as 
\begin{equation}
\eta(v) = \left( \frac{1}{2} \rho\, C_d\, A\, v^2 + C_{rr}\, m\, g \right) \quad \text{[N]}
\label{eq:energy_consumption}
\end{equation}

Equation~\eqref{eq:energy_consumption} reveals key design levers: mass \(m\), frontal area \(A\), drag coefficient \(C_d\), and rolling resistance \(C_{rr}\). This simplified model excludes transients like acceleration, braking, wind gusts, and idling, as well as the embodied energy of the vehicle and powertrain losses, which are addressed next.

\subsection{Driving Patterns: Beyond the Idealized Model}

Real-world driving involves multiple phases, each with distinct energy characteristics: cruising, accelerating, braking, and idling. Empirical studies (e.g., \cite{ma_real-world_2019}) provide typical phase distributions over urban trips:

\begin{figure}[h!]
    \centering
    \includegraphics[width=0.7\linewidth]{Figures/ch2_shareOfDrivingModeChina.jpg}
    \caption{Proportion of driving phases during urban operation (Source: \cite{ma_real-world_2019})}
    \label{fig:ch2proportiondrivingmode}
\end{figure}

\newpage 

Most trips in Europe are short, with 80\% under 10 km and 22 minutes \cite{donati_individual_2015}, reinforcing the relevance of frequent transient phases and vehicle warm-up times, especially for Internal Combustion Engine Car (ICE). Trips powered by human effort becomes a plausible benchmark for energy use over such durations.

\begin{figure}[h!]
    \centering
    \subfloat{
        \includegraphics[width=0.45\linewidth]{Figures/ch2_TripsLenghtFrequency.png}
        \label{fig:trips-length}
    }
    \hfill
    \subfloat{
        \includegraphics[width=0.45\linewidth]{Figures/ch2_TripsDurationFrequency.png}
        \label{fig:trips-duration}
    }
    \caption{Trip length and duration distributions in Europe (Source: \cite{donati_individual_2015})}
    \label{fig:trips-comparison}
\end{figure}

We define distinct phase efficiencies:

\begin{itemize}
    \item \textbf{Idling Efficiency:} Energy consumed per unit time when stationary. Typically:
    \begin{itemize}
        \item ICE: ~12 kW
        \item EV: $\leq$ 0.5 kW
    \end{itemize}

    \item \textbf{Braking/Deceleration Efficiency:} Energy recovered during braking.
    \begin{itemize}
        \item ICE: 0\%
        \item EV (regen): 50–70\% \cite{noauthor_regenerative_nodate}
    \end{itemize}

    \item \textbf{Acceleration Efficiency:} Energy transferred from tank/battery to kinetic motion.
    \begin{itemize}
        \item ICE: ~13\%
        \item EV: up to 80\% \cite{lohse-busch_ambient_2013}
    \end{itemize}
\end{itemize}

These phase-specific efficiencies reinforce the importance of designing for all driving modes, especially in urban environments characterized by frequent starts and stops.

\subsection{Embodied Energy and Material Impact: Why Mass Matters}

Although this work does not perform a full lifecycle analysis, it is important to acknowledge that manufacturing represents a non-negligible share of a vehicle’s total emissions especially for EVs with energy-intensive battery production. As grid carbon intensity decreases, manufacturing emissions become the limiting factor.

A low-mass, long-lived vehicle, built from materials with low embodied energy, offers a clear advantage in this regard.

\newpage 

\subsection{Parameters Affecting Efficiency}

From Eq.~\eqref{eq:energy_consumption}, we identify key parameters influencing operational energy efficiency:

\begin{itemize}
    \item Reduce \textbf{mass} \(m\) to minimize both rolling resistance and gravitational load.
    \item Minimize \textbf{frontal area} \(A\) and optimize \textbf{drag coefficient} \(C_d\).
    \item Lower the \textbf{rolling resistance coefficient} \(C_{rr}\) via tire selection and surface optimization.
    \item Maximize \textbf{powertrain efficiency} to reduce losses during acceleration and regenerative braking.
    \item Improve \textbf{idling efficiency}, especially critical for short, stop-start urban trips.
\end{itemize}

\subsection{Aerodynamic Optimization Through Form Factor}

Compact, narrow vehicle designs naturally limit frontal area \(A\). Reclined seating and tandem configurations can further reduce the product \(C_d A\), though this introduces challenges related to comfort and accessibility.

\begin{figure}[h!]
    \centering
    \subfloat{
        \includegraphics[width=0.47\linewidth]{Figures/ch3_seatingOptimisation.png}
    }
    \hfill
    \subfloat{
        \includegraphics[width=0.47\linewidth]{Figures/ch3_seatingOptimisationFront.png}
    }
    \caption{Effect of seat recline and tandem seating on frontal area}
    \label{fig:FrontaAreaGraphicsComparison}
\end{figure}

Digital augmentation (cameras, screens) could replace traditional visibility elements to further reduce \(A\), but may induce motion sickness due to visual-vestibular mismatches. A partially reclined posture with direct external visibility is a pragmatic compromise.

\begin{figure}[h!]
    \centering
    \includegraphics[width=0.8\linewidth]{Figures/ch4_frontComparisonVehicle.png}
    \caption{Frontal area comparison of car, leaning vehicle, proposed concept, and a standing person}
    \label{fig:frontal_comparison}
\end{figure}

Our proposed design incorporates a height-adjustable wheel system, enabling both dynamic tilt control and variable ingress/egress configurations combining aerodynamic form with accessibility.


\section{Design Considerations}

\subsection{Reducing Frontal Area and improving the coefficient of drag}

A typical passenger car offer five seats, while the average occupancy rate for private car remains well below two persons per car \cite{ceu_move_study_2022}.

To reduce the frontal area, we consider a smaller vehicle with one or two seat. Passengers would be one behind the other. This offers the benefit of both reducing the width of the vehicle while also allowing to create a more elongated envelope that improve the coefficient of drag. Furthermore, we can also reduce the effective used height of the vehicle by tilting the seats, allowing to further reduce the frontal area.

show graphically the frontal area of a car and the passenger inside. Front and side view. (graph SUV side + front and proposed seating side + front

show the proposed solution to reduce the space occupied with the tilted chair. show the lying position / human tuboid.

talk about ergonomic and the user FoV, limitation on height reduction (or deport viewing with set of problem like motion sickness)

Talk about ergonomics, how to get in and out
\subsection{Reducing mass and embodied energy}
why are car heavy ? what is energy intensive in a car ?

\subsection{Addressing Visibility and Safety Concerns in Traffic}

why trike and velomobile are not more mainstream (economics and  driving lower than anybody else, more ?)

\subsection{User comfort requirement}
(thermal, noise, water ingress, seating position)
study about what people need in car. we are we sensitive about when travelling / commuting


joystick -> poignet + bas que le coude

nuque -> 4-20°  -> max 15° depuis le 4° ->  total 30°

talk about ergonomy and the user FoV, limitation on height reduction (or deport viewing with set of problem like motion sickness)

https://www.epfl.ch/labs/chili/fr/chili/

\subsection{Narrow Track and Low Vehicle Geometry (literature review)}
Acknowledge existing solution
\section{Innovative Design Concept}

\subsection{Elevating the Vehicle Without Increasing Frontal Area}

from previous chapter, want to lower frontal area while keeping user eyes high from the ground

how can we do so on a kinematic standpoint and what synergies can we gain from coupling the steering to the leaning ?\\

*** steering approach\\
-> skid steer\\
-> ackerman\\
-> swerve wheel + crab steering\\
-> 2 wheel lean to steer (trail + no trail ?)\\
-> other\\
*** geometry and kinematic\\
-> vertical actuator\\
-> arm along vehciule\\
-> arm sideways vehicule\\
-> others ?\\

\subsection{mass reduction}
can we afford high performance material ? do they make sense ? what material can we use to keep mass and carbon imprint low

\subsection{Benefits for Vertical Parking and Accessibility}
talk about the advantage of the selected design, how it could park vertically and what good it would do.\\

talk about the stability while boarding and exiting, how the vehciule can help to make it a "normal chair" posture then reclining\\

\subsection{Literature review of existing concept}

\subsection{Integration with Public Infrastructure (e.g., Trains)}
\section{Evaluation of Existing Concepts}

\subsection{Limitations of Cycle-Based Vehicles in Urban Use}
The table above compares key performance and user experience parameters of E-bikes and velomobiles, grouped into three categories: efficiency, urban usability, and human factors. While both vehicle types demonstrate strong performance in terms of drivetrain efficiency, low aerodynamic drag, and compact form factors suitable for urban environments, they score notably lower on human-centered criteria. These include aspects such as protection, comfort, perceived visibility, and ease of access factors that play a significant role in perceived usability and day-to-day convenience.

Although this analysis does not establish causation, the correlation between low user adoption rates and poor performance on human-centered parameters is striking. It is the author's view that even without a guaranteed improvement in acceptance, addressing these user experience limitations is a worthwhile design direction that could help bridge the gap between technical efficiency and real-world desirability.

\begin{table}[h!]
\centering
\caption{Limitations of Cycle-Based Vehicles in Urban Use}
\begin{tabular}{lcc}
\toprule
\textbf{Parameter} & \textbf{E-Bike} & \textbf{Velomobile} \\
\midrule
\multicolumn{3}{l}{\textbf{Section 1: Efficiency}} \\
Drive train efficiency    & \cellcolor{LightGreen}$\geq70\%$ & \cellcolor{LightGreen}$\geq70\%$ \\
$C_d$ (drag coefficient)     & \cellcolor{LightRed}1.1        & \cellcolor{LightGreen}0.15 \\
Frontal area                 & \cellcolor{LightGreen}0.5 m²   & \cellcolor{LightGreen}~0.35 m² \\
Mass $m$         & \cellcolor{LightGreen} $\approx$85 kg & \cellcolor{LightGreen} $\approx$110 kg \\
$C_{rr}$ (rolling resistance)& \cellcolor{LightGreen}0.004    & \cellcolor{LightGreen}0.004  \\
\midrule
\multicolumn{3}{l}{\textbf{Section 2: Urban Elements}} \\
Width            & \cellcolor{LightGreen} $\approx$ 50 cm & \cellcolor{LightGreen}$\approx$ 70 cm \\
Parking space requirement    & \cellcolor{LightGreen}$\approx$ 1 m²& \cellcolor{LightGreen}$\approx$ 1.5 m²\\
Speed                        & \cellcolor{LightOrange}45 kmh     & \cellcolor{LightOrange}45 kmh \\
Power-to-weight (continuous) & \cellcolor{LightOrange}13 W/kg        & \cellcolor{LightOrange}10 W/kg\\
Power-to-weight (peak)       & \cellcolor{LightOrange}16 W/kg  & \cellcolor{LightOrange}13 W/kg\\
\midrule
\multicolumn{3}{l}{\textbf{Section 3: Human Elements}} \\
Feeling visible / high road  & \cellcolor{LightGreen}Good     & \cellcolor{LightRed}bad \\
Feeling protected            & \cellcolor{LightRed}Poor       & \cellcolor{LightOrange}Fair \\
Physical effort              & \cellcolor{LightOrange}Medium  & \cellcolor{LightOrange}Medium \\
Cargo \& passenger capacity  & \cellcolor{LightRed}Low        & \cellcolor{LightOrange}Fair \\
Weather / thermal protection & \cellcolor{LightRed}Minimal    & \cellcolor{LightGreen}Good \\
Privacy / comfort            & \cellcolor{LightRed}Poor       & \cellcolor{LightOrange}Fair \\
Ease of entry / exit         & \cellcolor{LightOrange}Fair     & \cellcolor{LightRed}Difficult \\
\bottomrule
\end{tabular}
\end{table}

\newpage 

\subsection{Limitations of Car for Urban Use}

The table bellow highlights the performance of internal combustion engine (ICE) and electric cars across efficiency, urban suitability, and human-centered aspects. While both types of cars excel in comfort, safety, and user convenience, they face significant limitations in terms of efficiency and urban compatibility. ICE cars, in particular, suffer from very low drivetrain efficiency, while both vehicle types share drawbacks like high mass, large frontal area, and substantial parking space requirements. These factors not only reduce energy efficiency but also exacerbate urban challenges such as traffic congestion, noise pollution, and the growing space scarcity in dense cities.

Addressing these inefficiencies is critical both for the urgent need to reduce $CO_2$ emissions, and the increasing strain on urban infrastructure. Optimizing for smaller, lighter, and more space-efficient vehicles could significantly alleviate traffic jams and parking shortages, while also supporting broader sustainability goals. Therefore, it should be a key focus of future mobility solutions, but as cycle based vehicle addoption show, this should not be done at the expense of user comfort and practicality.

\begin{table}[h!]
\centering\caption{Limitations of Cycle-Based Vehicles in Urban Use}
\begin{tabular}{lcc}
\toprule
\textbf{Parameter} & \textbf{ICE Car} & \textbf{Electric Car} \\
\midrule
\multicolumn{3}{l}{\textbf{Section 1: Efficiency}} \\
Drive train efficiency    & \cellcolor{LightRed}$\leq30\%$ & \cellcolor{LightGreen}$\geq80\%$ \\
$C_d$ (drag coefficient)     & \cellcolor{LightRed} 0.3         & \cellcolor{LightOrange}0.2 \\
Frontal area                 & \cellcolor{LightRed}~2.3 m²   & \cellcolor{LightRed}~2.3 m² \\
Mass $m$         & \cellcolor{LightRed} $\approx$1100 kg & \cellcolor{LightRed} $\approx$1300 kg \\
$C_{rr}$ (rolling resistance)& \cellcolor{LightOrange}0.01    & \cellcolor{LightOrange}0.01  \\
\midrule
\multicolumn{3}{l}{\textbf{Section 2: Urban Elements}} \\
Width            & \cellcolor{LightRed} $\approx$ 180 cm & \cellcolor{LightRed}$\approx$ 180 cm \\
Parking space requirement    & \cellcolor{LightRed}$\approx$ 12 m²& \cellcolor{LightRed}$\approx$ 12 m²\\
Speed                        & \cellcolor{LightGreen}120 kmh     & \cellcolor{LightGreen}120 kmh \\
Power-to-weight (continuous) & \cellcolor{LightGreen}50 W/kg        & \cellcolor{LightGreen}60 W/kg\\
Power-to-weight (peak)       & \cellcolor{LightGreen}50 W/kg  & \cellcolor{LightGreen}120 W/kg\\
\midrule
\multicolumn{3}{l}{\textbf{Section 3: Human Elements}} \\
Feeling visible / high road  & \cellcolor{LightGreen}Good     & \cellcolor{LightGreen}Good \\
Feeling protected            & \cellcolor{LightGreen}Good       & \cellcolor{LightGreen}Good \\
Physical effort              & \cellcolor{LightGreen}None  & \cellcolor{LightGreen}None \\
Cargo \& passenger capacity  & \cellcolor{LightGreen}Good        & \cellcolor{LightGreen}Good \\
Weather / thermal protection & \cellcolor{LightGreen}Good    & \cellcolor{LightGreen}Good \\
Privacy / comfort            & \cellcolor{LightGreen}Good       & \cellcolor{LightGreen}Good \\
Ease of entry / exit         & \cellcolor{LightGreen}Good     & \cellcolor{LightGreen}Good \\
\bottomrule
\end{tabular}
\end{table}

\newpage

\subsection{Narrow Track Vehicles: Literature Overview and Modeling Considerations}

As detailed in the following chapter, our proposed solution to the shortcomings of conventional urban mobility takes the form of a \textbf{Narrow Track Vehicle (NTV)}. These vehicles, which actively lean into turns, aim to combine the compact footprint and energy efficiency of two-wheelers with the stability and enclosure benefits of multi-wheeled platforms.

\subsection*{Historical Background and Evolution of NTV Research}

Tilting vehicles have been studied since at least the 1950s. However, significant progress in their design and implementation only emerged in the 1990s, driven by advances in control theory and numerical simulation. Increasing urban density and demand for compact mobility solutions redirected attention toward narrow-track configurations.

Today, these vehicles are referred to under various terms such as \textit{tilting three-wheelers}, \textit{man-wide vehicles}, or more generally, \textit{Narrow Track Vehicles (NTVs)}. Their key characteristic is the necessity to actively manage lateral stability during dynamic maneuvers. Without tilting, such narrow vehicles are prone to roll instability and fallover during turns. Early NTV attempts frequently failed due to transient instabilities, sudden and dangerous behaviors triggered during abrupt direction changes or when encountering irregular road conditions. Such behaviors cannot be predicted by simplified static or kinematic models.

\subsection*{The Need for Multibody Simulation and Control}

Accurate modeling of NTV dynamics requires multibody simulation. As demonstrated by Docquier \cite{docquier_dynamic_nodate}, high-fidelity dynamic modeling is essential to capture nonlinear and transient phenomena such as tipping under rapid maneuvers, delayed responses, and steering-induced oscillations. Simplified planar models or bicycle-model approximations fall short when analyzing these effects and are insufficient for control design and geometry comparison.

Stability and handling in tilting NTVs are achieved through active control of lean angle and steering angle. Unlike conventional four-wheelers, which are statically and dynamically stable, NTVs behave more like bicycles or motorcycles. Their dynamics involve coupling between steering and roll motion and this tend to require advanced control scheme.

\newpage 

\subsection*{Control Strategies and User Interaction}

Two dominant control strategies can be found in the literature:

\begin{itemize}
    \item \textbf{Indirect tilt control}, where steering input generates roll motion via inertial and tire forces—this is typical of motorcycles and some passive or semi-active tilting trikes. The user must perform a countersteering maneuver (i.e., momentarily steering in the opposite direction) to initiate the lean and turn.
    
    \item \textbf{Direct tilt control}, where the lean angle is explicitly actuated (e.g., via hydraulic actuators or linear motors), and steering is either coupled or controlled in parallel. This allows the vehicle to follow a commanded trajectory without requiring complex rider input and may feel more natural in enclosed or drive-by-wire platforms.
\end{itemize}

Indirect control strategies rely heavily on the rider's skill and experience, while direct control strategies increase system complexity and demand robust sensor fusion, trajectory planning, and closed-loop control. Both approaches are still actively studied.

\subsection*{Reliability and Safety Considerations}

As NTVs rely on active control to ensure lateral stability, their reliability under fault conditions is a major concern. In the event of actuator failure, power loss, or sensor dropout, the vehicle could lose its ability to stabilize itself and become hazardous. 

Some concepts address this by incorporating passive fallback modes, such as mechanically locking the tilt mechanism, reverting to a stable tripod configuration, or gradually reducing speed to regain static stability. Others explore redundant actuation or degraded-mode control schemes. However, few studies systematically address fault detection, diagnosis, and safe-state transitions. More research is needed to ensure that tilting NTVs can handle real-world disturbances and hardware failures without endangering occupants or surrounding traffic.

\subsection*{Model Simplification for Preliminary Control Design}

To avoid having to design a dynamic controller as this is very time consumming and would justify it's own semester or master project, we simplify the problem by demonstrating the existence of a steady-state controller for straight-line motion and constant-radius turns using simple tuned PID. While this does not capture transient or disturbance behavior, it will allow to compute some metrics to start comparing the designs.

\newpage 

\subsection*{Performance Metrics for Dynamic Comparison}

Once a basic control scheme is in place, it becomes possible to assess and compare different vehicle architectures based on dynamic performance. However, due to the simplified nature of our current model and the lack of a full feedback controller, only a subset of these metrics can be meaningfully evaluated at this stage.

The following categories illustrate typical metrics used in the literature to evaluate tilting NTVs:

\begin{itemize}

    \item \textbf{existence of steady state stability:} For  agiven vehicle speed and heading, we show that by tunning a PID the vehicle can be stabilised. It's not a strong proof by any stretch of imagination but it help getting a feeling of what design can work and which cannot.

    
    \item \textbf{Roll Angle vs. Turn Radius:} For a given vehicle speed, the required steady-state roll angle to maintain a stable corner is a function of geometry and mass distribution. This can be derived analytically and used to evaluate how much lean is needed in typical maneuvers This will not be done here as is depend of the final design and have no value at this stage.
    
    \item \textbf{Steady-State Control Effort:} The required steering or tilt input torque under steady cornering provides insight into actuator sizing and energy efficiency. These values are geometry-dependent and can be computed under the assumption of ideal controllers.
    
    \item \textbf{Cornering Stability Margin:} This refers to the difference between the equilibrium lean angle and the critical tipping angle. It provides a static safety margin but does not account for transient or delayed responses.
    
    \item \textbf{Transient Recovery Time and Overshoot:} These metrics, which assess how quickly and smoothly a system responds to changes in command or disturbances, require a full dynamic controller to simulate and are therefore not addressed in this initial study. However, they are essential for understanding rider comfort and system robustness in a final implementation.
    
    \item \textbf{Disturbance Rejection and Fault Resilience:} Evaluating how the system responds to crosswinds, road bumps, or sensor noise also necessitates a closed-loop controller with disturbance modeling. These factors play a critical role in real-world safety but cannot be fully quantified at this stage.
    
    \item \textbf{Natural Frequency and Damping of Lean Oscillations:} These can provide insight into the passive dynamics of the vehicle frame and its propensity for wobble or instability. This will not be measured due to time constraint.
\end{itemize}

In summary, while many of the most informative performance metrics require a full closed-loop control system to simulate realistically, geometry-dependent indicators like roll angle requirements and equilibrium tipping margins can still offer valuable early insight. These form the basis for comparative studies between NTV designs prior to full control implementation. Once a dynamic controller is available, more complete evaluation including maneuverability, ride comfort, and fault handling will be possible.

\section{System integration}

\subsection{Trade-offs Between Design Choices}

\subsection{power, speed gravity, acceleration, braking}

from chapter 2, how much accelerating do we need ? (look at comfortable acceleration in traffic to keep it smooth at a red light for instance. also look at hill climbing what slope can be expected and at what speed\\)

how much regenerative braking ? (look at braking data, aim to recover energy in most case as full emergency braking are very rare and most braking is under 0.3g) \\

what power for the motor ? (battery asymmetry charge discharge, size for climbing up and acceleration in traffic)\\5

what battery size ? look at daily km, 99\% cutoff and previous efficiency\\
(likely 1kwh, 250 km range, 10kg)
impact on range, climbing hill, user acceptance \\

\subsection{System review}
how do we compare against other solutions ?\\
key performance indicator ? \\

\section{Static Behavior and control margin}

Computing the control margin is essential to understand how much the effective resultant vector at the vehicle’s center of mass (CoM) influenced by gravity, road slope, and centrifugal forces  can vary before the vehicle tips over. This margin quantifies the allowable tilt angle range within which the vehicle remains stable. The leaning controller continuously adjusts the vehicle’s tilt to keep it safely within this window, maintaining enough margin on each side so it has sufficient time to react to disturbances. Properly sizing this stability margin is critical because it directly informs the required response speed of the actuator responsible for leaning control, ensuring the vehicle remains balanced under dynamic conditions.

The vehicle will tip over when:

\begin{figure}[h!]
    \centering
    \includegraphics[width=0.55\linewidth]{Figures/ch7_tippingPoint.png}
    \caption{Simplified model of vehicle tipping point}
    \label{fig:vehicleTipping}
\end{figure}
$$
\tan\left( \arctan\left(\frac{\delta L}{a}\right) + \theta \right) > \frac{a \left(M_t + 4m_f\right)}{2 \left( (M_t + 2m_f)L - m_f \delta L \right)}
$$


Parameter Definitions

\begin{itemize}
  \item $a$: Horizontal distance between the legs (along the tipping direction).
  \item $\delta L$: Height difference between the long and short legs.
  \item $L$: Length of the longer legs.
  \item $M_t$: Mass of the vehicle cabin.
  \item $m_f$: Mass of each leg+wheel (assumed identical for all four legs).
  \item $\theta$: Angle between the resulting vector on the CoM and the normal of the plane made by the top of the leg, in the tipping direction. This help to model the gravity if the road is slopped and the centrifugal acceleration:
    \begin{itemize}
      \item $\theta > 0$: tipping is more likely.
      \item $\theta < 0$: tipping is resisted.
    \end{itemize}
\end{itemize}
we can then compute the stability margin on each side with the following.
\begin{align*}
\theta_{\text{critical,left}} &= \arctan\left(\frac{a \left(M_t + 4m_f\right)}{2 \left( (M_t + 2m_f)L - m_f \delta L \right)}\right) - \arctan\left(\frac{\delta L}{a}\right)\\
\theta_{\text{critical,right}} &= \arctan\left(-\frac{a \left(M_t + 4m_f\right)}{2 \left( (M_t + 2m_f)L - m_f \delta L \right)}\right) + \arctan\left(\frac{\delta L}{a}\right) \\
\\
\text{Stability margin on left side} &= \theta_{\text{critical,left}} - \theta \\
\text{Stability margin on right side} &= \theta - \theta_{\text{critical,right}}
\end{align*}
Without surprise, to maximize the stability margin, we should make the vehicle as wide as possible and keep the height of the center of mass as low as possible. But as always it's a matter of tradeoff, the final stability margin will be given by the controller, and thus we can tune these parameters to have just what is necessary to avoid losing the benefit of having a NTV. If the need arose we could make the leg deploy sideway instead of along the body to temporarily increase the width of the vehicle and thus when the system leans the margin also increase. But this is mechanically more complicated as it requires keeping both wheel parallel if we want to avoid adding more non-linear coupling which make the controller more complicated as we wheel also need to compensate for the shift of the contact patch on the wheel.
\section{Economic and Practical Viability}

\subsection{Energy Efficiency Comparison with Conventional Cars}

\subsection{Cost-Benefit Analysis of the Proposed System}

\subsection{Scalability and Market Potential}



\section{Implementation Scenarios}

\subsection{Integration into Urban Environments}
if such vehicule was deployed, how would traffic look like inside city ? 

\subsection{Multi-Modal Transport Efficiency with Train Onboarding}
what would happen if we took the train with our "bike++" to coverage long distance ?

\subsection{Real-World Use Cases and Potential Impact}
\section{Appendix}

\subsection{Daily kilometers average}
(on average, less than 5 trip per day)
\begin{figure}[h!]
    \centering
    \subfloat{
        \includegraphics[width=0.97\linewidth]{Figures/ch2_DailyLenghtFrequencyPlot.png}
        \label{fig:daily-length}
    }
    \hfill
    \caption{Relative frequency (bar) and cumulative (red, solid) distributions of daily mobility distance (from Donati 2015\cite{donati_individual_2015})}
    \label{fig:daily-lenght-travel}
\end{figure}

\newpage 

\subsection{logbook}

\begin{itemize}[leftmargin=1.5cm,label={}]
    \item[\textbf{2025-02-20}] Creation of this report, rough layout of the idea in chapter
    
    \item[\textbf{2025-02-25}] Longitudinal kinematic model, research about markov modelling and cycle
    
    \item[\textbf{2025-02-26}] dynamic simulation creation, implemented a basic simulation in pybullet with keybaord navigation and parametric surface + wheel rolling down
    
    \item[\textbf{2025-02-27}] refining efficiency to take into account whole cycle, reading paper
    
    \item[\textbf{2025-02-28}] Writing Idea about improving efficiency trough identified parameters

    \item[\textbf{2025-03-04}] Adding graph about car occupancy and real world driving needs + reading paper

    \item[\textbf{2025-03-06}] Brainstorming kinematic, reading PHd thesis on narrow track

    \item[\textbf{2025-03-09}] URDF is a PAIN to implement to due lack of tool, trying to define a kinematic model in python

    \item[\textbf{2025-03-11}] reworking part of chapter 2

    \item[\textbf{2025-03-17}] reading paper about cars and bike and limiting factor to adoption

    \item[\textbf{2025-03-18}] keep reading paper, also about narrow track vehicle

    \item[\textbf{2025-03-20}] trying to structure what I learned from the papers

    \item[\textbf{2025-03-21}] defining the know unknown about car adoption factor (human side)

    \item[\textbf{2025-03-24}] Clarifying the project goal 

    \item[\textbf{2025-03-27}] draw figure, added citation

    \item[\textbf{2025-03-28}] Manual definition is an epic failure... Two way (createMultiBody -> Works but hard to read), Create body and add constraint don't work (rotation constraint crash the simulation.)

    \item[\textbf{2025-04-09}] Realized that createMultibody is more stable but has a "ghost" friction problem, Might as well use URDF then.

    \item[\textbf{2025-04-10}] learning URDF and how the origin is defined, 

    \item[\textbf{2025-05-25}] redefined most variant as URDF and how to simulate them. Need spend time to implement basic controller like PID to show steady state stability and control margin

    \item[\textbf{2025-05-28}] Meeting, dry run presentation  and shown current status.

    \item[\textbf{2025-05-31}] Report restructuring, writing

    \item[\textbf{2025-06-02}] Chapter 4, 5, 6 writing
\end{itemize}


% Add appendices here, if there are any

\newpage
\bibliographystyle{plain} 
\bibliography{References}

\end{document}
